% !TEX root = ../The-Haptic-Printer.tex
%
\chapter{Introduction}
\label{sec:intro}

The haptic technology enables human computer interaction in a new way compared to previous existent interaction techniques. Haptics gives us a new dimension of interaction technique without any physical contact between the user and the device. This contactless haptic technologies give rise to various kind of applications in real life scenarios. One prominent among them is the mid air user interfaces, where the use of hand and finger movements are used to control the interface. Ultrasound haptic technology \cite{iwamoto2008non} is one of the example of contactless haptic technology where ultrasound waves are emitted by the device and coincide at a particular point in space to give haptic sensations. 


\section{About Ultrahaptics}
\label{sec:intro:about ultrahaptics}


The human skin has mechanoreceptors which are sensory cells that convert mechanical forces into nerve excitation. When the sound waves make contact with the skin, it gives a tactile sensation for the user. In order to produce such ultrasound waves we make use of a device manufactured by Ultraleap \cite{ul} called as Ultrahaptics.	The device emits ultrasound from its phased array which propagates as a wave with a frequency higher than the limit of human hearing. Each emitter in the device emits the wave and every wave emitted coincide at a point in the air. This coincidence of the waves at a point is called a focal point and this focal point when directed towards a user palm creates a pressure for the user and the user can perceive it as a tactile sensation. The coincidence of waves at a focal point, strength and intensity of each transducer is handled by the Ultrahaptics device itself by using various solver algorithms.

In order to render custom shapes we need to explore various techniques of rendering and experiment with different parameters of the device. The different parameters available in the device are intensity, frequency, multiple focal points and the different techniques of rendering are explained in the next section of literature survey. 

\section{Project objective}
\label{sec:intro:objective}

Using the concept of mid air interfaces, we can explore and build many applications in real world. In the recent times due to the COVID-19 pandemic, one prominent use case would be to build contactless interfaces in the public space. In order to achieve this we need to render custom shapes to the user in order to make user aware of different kinds of sensation.
In this project, our main focus is to render custom shapes on user's palm using different kinds of inputs. Here, different kinds of inputs means different various formats like CSV (comma-separated values), SVG (Scalable Vector Graphics), user drawing and Leap Motion. We process all these kind of inputs and render the dynamic shapes using the Ultrahaptics device. 

